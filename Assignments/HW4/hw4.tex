\documentclass{article}
\usepackage{graphicx} % Required for inserting images
\usepackage{amsmath}
\usepackage{amssymb}
\usepackage{enumerate}
\usepackage[margin=1in]{geometry}
\usepackage{tcolorbox}
\usepackage{bm}
\usepackage{graphicx}


\newcommand{\mydivider}{\vspace{1em}\hrule\vspace{1em}}
\newcommand{\mypart}[2]{\noindent{\textbf{[#1] point(s) --- #2:}}}
\newcommand{\programming}{\textcolor{blue}{This is a programming exercise. See \texttt{hw4.ipynb}}}



%%% BEGIN MACROS
% type your macros here
%%% END MACROS


\title{CS 577 --- Deep Learning --- Homework 4}
\author{}
\date{}

\begin{document}

\maketitle
\textbf{Read these instructions carefully}:
\begin{itemize}
  \item
In the \LaTeX~source code, type your answer in between
``\verb|%%% BEGIN ANSWER|'' and
``\verb|%%% END ANSWER|''.
        For advanced \LaTeX users,
        you can use your custom macros if you wish
        by placing them
        between
``\verb|%%% BEGIN MACROS|'' and
``\verb|%%% END MACROS|'' in the header.
        Do not modify anything else.




  \item Turn in both your \verb|.tex| file and the generated \verb|.pdf| file.
\end{itemize}



\section{Backpropagation}

\mypart{5}{part a}

Answer the question in Section 1.3.7 of
\texttt{backpropagation.pdf}:
How can we calculate
\(\frac{\partial f}{\partial z_{4}}(x,y)
\)
given
correct value of
\(
\frac{\partial f}{\partial z_{5}}(x,y)
\)?
\begin{tcolorbox}
\textbf{Answer: }

%%% BEGIN ANSWER
Using the chain rule, we can calculate the derivative of \(f\) with respect to \(z_{4}\) as follows:
$$\frac{\partial f}{\partial z_{4}} = \frac{\partial f}{\partial z_{5}} (x,y) \frac{\partial z_{5}}{\partial z_{4}} $$
%%% END ANSWER

\end{tcolorbox}

\mypart{5}{part b}

Answer the question in Section 1.3.8 of \texttt{backpropagation.pdf}: What is currently stored in \texttt{z3.grad} right before \texttt{z4.\_backward()} is called?

\begin{tcolorbox}
\textbf{Answer: }

%%% BEGIN ANSWER
\(\frac{\partial f}{\partial x_3}\) is stored in z3.grad . Using the chain rule, we can calculate the derivative of \(f\) with respect to \(z_{3}\) as follows:
$$\frac{\partial f}{\partial x_3} = \frac{\partial f}{\partial x_8}(x,y)\frac{\partial x_8}{\partial x_3}$$
We know that \(\frac{\partial f}{\partial x_8}(x,y)\) is 1, as $z_8 = f(x,y)$ and $\frac{\partial x_8}{\partial x_3}$ is $ \frac{\partial (x_7 * x_3)}{\partial z_3}$, so:
$$\frac{\partial f}{\partial x_3} = \frac{\partial (x_7 * x_3)}{\partial z_3}$$ 
Using the product rule and knowing the derivative of \(z_3\) with respect to \(z_3\) is 1, we get:
$$\frac{\partial f}{\partial x_3} = z_7 \frac{\partial z_3}{\partial z_3} + z_3 \frac{\partial z_7}{\partial z_3} = z_7 + z_3 \frac{\partial z_7}{\partial z_3}$$
The first term is $z_7$ and the second term is  $z_3 \frac{\partial z_7}{\partial z_3}$. This last term has not been compute because it depends on the backward pass through $z_4$.
So, the value of \texttt{z3.grad} stored before calling \texttt{z4.\_backward()} is $z_7$. The second term will be added once backpropagation processes earlier nodes.
%%% END ANSWER

\end{tcolorbox}


\section{Gradient descent with \texttt{ag.Scalar}}

\mypart{10}{part a}  \programming

\section{Transformer with \texttt{ag.Scalar}}
\mypart{Bonus 20}{part a}  \programming

\end{document}
